%\documentclass[10pt]{iopart}
\documentclass{article}
\usepackage{geometry}
\usepackage[nottoc]{tocbibind}
\usepackage{graphicx}
\usepackage{subcaption}
\usepackage{amssymb}
\usepackage{amsmath}
\usepackage{amsthm}
\usepackage{slashed}
\usepackage{bbold}
\usepackage{bm}
\usepackage{hyperref}
\usepackage{listings}
\usepackage[percent]{overpic}
\usepackage{cite}
\usepackage{yfonts}
\usepackage{physics}
\usepackage{todonotes}
\usepackage{caption}
\usepackage{subcaption}
\usepackage{tcolorbox}
%\usepackage{axodraw4j} %for Feynman diagrams

%Uncomment next line if AMS fonts required
%\usepackage{iopams}
\numberwithin{equation}{section} %allows numbering of equations to include section number
\usepackage[utf8x]{inputenc}
\geometry{a4paper, total={6in, 8in}}

\hypersetup{
    colorlinks=true,
    linkcolor=blue,
    filecolor=magenta,      
    urlcolor=red,
    citecolor=purple   
}

\allowdisplaybreaks

\begin{document}


\begin{titlepage}
\begin{center}
\vspace*{1.5cm}

{\large Let's Chat About \\ \vspace{0.25cm}\LARGE \textit{Quantum Mechanics}}\\
\vspace{3cm}
Andrew Zhen Ning Yong\\
\vspace{0.25cm}

Department of Physics \& Astronomy\\
University of Edinburgh

\vspace{7cm}


\end{center}
\end{titlepage}

\tableofcontents

\newpage

\section{Preamble}
First, something of an explanation and a disclaimer regarding this document.

This is my attempt at writing an informal discussion based on the content in the PHYS09053/09051 Quantum Mechanics course by C. Stock and A. Huxley. Needless to say, at best, this document should be read \textit{in complement with} the lecture notes provided, rather than instead of. 

One final thing before we start: I am prone to careless mistakes, especially where signs and numerical factors are concerned. So, if you notice anything suspicious with the expressions detailed below, please send me an email or send a pull request to my git repository.\\

\noindent Now let's chat about Quantum Mechanics.

\section{An Electron and Its Proton Pal}

In my first attempt, I will whizz through a recap on the hydrogen atom, reminding us the quantum numbers $n,l,s,m_l$ and $m_s$. This section may be expanded, but for now it serves the purpose of bringing us to the spin-orbit interaction.

Let us begin by placing an electron in the presence of a proton. Just like that, we have formed our very first atom - the hydrogen! Since the electron (by convention) has an electric charge of -1, and the proton +1, they both experience a Coulomb attraction due to each other. We can express the Hamiltonian of this system (in spherical polar coordinates) as

\begin{equation}
\hat{H}_0=-\frac{\hbar^2}{2\mu}\nabla^2 + \frac{e^2}{4\pi\epsilon r},
\label{h atom hamiltonian}
\end{equation}
where $\mu=\frac{m_eM_P}{m_e+M_P}$ is the reduced mass and the $\nabla^2$ is the double derivative operator in spherical polar coordinates. As you recall, it has a terribly cumbersome form - one which I will spare you from since it is not relevant at the moment.

When we solve the Schr\"odinger equation, $H_0\psi=E\psi$, we get a wavefunction that has quantum number $n, l$ and $m_l$. Here I will list a useful relation between this set of quantum numbers:

\begin{equation}
\begin{split}
n&=1,2,3,\dots \\
l&=0,1,\dots n-1,\\
m_l&=-l,-l+1,\dots,l-1,l.
\end{split}
\end{equation}
They exist here as a useful cheat sheet, but I encourage you to investigate why that is the case. 

This wavefunction is an eigenfunction of $H_0$, and the eigenvalues are 

\begin{equation}
E_n=-\frac{\mu e^4}{(4\pi\epsilon_0)^22\hbar^2n^2}=-\frac{13.6\text{eV}}{n^2}.
\label{h atom energy}
\end{equation}
Here, the energy (eigenvalue) is negative, which implies we have an attractive potential, thereby confirming that our results indeed correspond to a system with opposite charge. Also, we see that in cases where $n>1$, different $l$'s amd $m_l$'s (corresponding to different wavefunctions) give the same energy (since it only depends on $n$). We call these distinct states that give the same energy \textit{degenerate states}.


Including spin, where electrons are known to have $s=1/2$, we double the degeneracy to each energy level as $2n^2$. Let's make things more concrete now. To be concise, we switch to braket notation and talk in operator language. The Schr\"odinger equation looks like so

\begin{equation}
\hat{H}_0\ket{n,l,s,m_l,m_s} = E_n\ket{n,l,s,m_l,m_s}.
\label{h atom schrodinger equation}
\end{equation}
Each of the angular momentum quantum numbers correspond to an operator that commutes with the Hamiltonian operator, \textit{ie} $[\mathcal{O},H]=0$ for $\mathcal{O}\in\{\hat{l}^2,\hat{s}^2,\hat{l}_z,\hat{s}_z\}$. In the language of quantum mechanics, we call the quantum numbers corresponding to operators that commute with the Hamiltonian \textit{good quantum numbers}.

Look at Equation \ref{h atom schrodinger equation}. We have many degenerate states that correspond to the same energy $E_n$. Take an example of $n=2, s=1/2$. First, we have $m_s=\pm 1/2$. At $n=2$, we can have $l=0,1$ and therefore $m_l=\{-1,0,0,1\}$ (zero occurring once for each $l$). Together there are $2\times 4=8$ states that correspond to $E_2$.

I wonder if it is possible to lift some these degeneracies? (Yes.)

\subsection{Spin-Orbit Interaction}

Let's think about this for a little. Looking at $\hat{H}_0$ in Equation 
\ref{h atom hamiltonian}, this operator only describes the Coulomb attraction between the electron and the proton. Clearly, while a good starting point, $H_0$ does not capture the full dynamics of a hydrogen atom. Let's think of something it may have missed out.

To revert back to a classical picture (which is wrong, though a useful pictorial aide), recall that the electron is `orbiting' about the proton. Now, this electron in `circular motion' around the proton is effectively a current (composed of one mighty electron) going in a loop. Thus, Ampere tells us that a magnetic field will be generated by this current. This magnetic field, \textit{in turn}, interacts with the spin magnetic moment of the electron. This self interaction between two properties of the electron - the orbital angular momentum it experiences looping round the proton and its spin angular momentum - give rise to the \textit{spin-orbit} interaction, thus the title of this subsection.

We can write a Hamiltonian that captures this effect in the following manner:

\begin{equation}
\hat{H}_{SO}=\xi(\hat{r})\hat{l}\cdot\hat{s}.
\end{equation}
Here, $\xi$ is just a function of the distance to the nucleus. Its exact form is not needed here, for the main players in this game are the electron's orbital and spin angular momentum operators, $\hat{l}$ and $\hat{s}$ respectively. Our new and improved full Hamiltonian thus has the following form

\begin{equation}
\hat{H}= \hat{H}_0 + \hat{H}_{SO} = -\frac{\hbar^2}{2\mu}\hat{\nabla}^2 + \frac{e^2}{4\pi\epsilon \hat{r}} +  \xi(\hat{r})\hat{l}\cdot\hat{s},
\end{equation}
where all our dynamic variables wear a hat because they are operators in our braket notation.

So we have a new Hamiltonian to consider. What might the energy be when we include spin-orbit interaction? Put it a different way, what might the energy \textit{shift} from Equation \ref{h atom energy} be due to the spin-orbit interaction? This screams perturbation theory.

There is a minor hurdle we must overcome beforehand. Previously, the operators $\hat{l}^2,\hat{s}^2,\hat{l}_z,\hat{s}_z$ commute with $\hat{H}_0$. In our tutorial, we worked out that $[\hat{l}_z,\hat{H}_{SO}]\neq 0$ and similarly for $\hat{s_z}$, but $[\hat{l}_z+\hat{s_z},\hat{H}_{SO}]= 0$. In the previous case, $m_l$ and $m_s$ were good quantum numbers, but now when we include spin-orbit interaction, they are no longer the bee knees. So, we must replace these two quantum numbers with another two that \textit{also} commutes with $\hat{H}_0$.

We have seen that the combination $m_l+m_s$ is a good quantum number since $[\hat{l}_z+\hat{s}_z,\hat{H}_{SO}]=[\hat{l}_z+\hat{s}_z,\hat{H}_{0}]=0.$ We need one more candidate quantum number. A reasonable guess, looking at the $\hat{l}\cdot\hat{s}$ structure of $\hat{H}_{SO}$, is $(\hat{l}+\hat{s})^2$ since $[\hat{l}^2,\hat{l}_i\hat{s}_i]=0$ for $i\in\{x,y,z\}$ and similarly for $\hat{s}$. 

Thus, let us construct the following operators:

\begin{equation}
\begin{split}
\hat{j}&=\hat{l}+\hat{s} \quad \text{s.t}\quad  \hat{j}\ket{j,m_j} = \hbar^2j(j+1)\ket{j,m_j}, \\
\hat{j}_z&=\hat{l}_z+\hat{s}_z \quad \text{s.t} \quad \hat{j}_z\ket{j,m_j} = m_j\hbar\ket{j,m_j}
\end{split}
\end{equation}
with $-j\leq m_j \leq j$ as in the case with the orbital and spin angular momentum. Then, the new eigenstate for our full Hamiltonian becomes
\begin{equation}
\ket{n,l,s,m_l,m_s}\rightarrow \ket{n,l,s,j,m_j},
\end{equation}
where the good quantum numbers are now $\{n,l,s,j,m_j\}$.

Previously, we found the bounds for $l$ quantum number from solving the hydrogen atom Schr\"odinger equation. We now need a way to determine numerically what values $j$ (and therefore $m_j$) can take. To that end, we must turn to the angular momentum addition theorem. \\




\begin{tcolorbox}[colback=pink]
Hamiltonian including Spin-Orbit Interaction
\begin{equation}
\hat{H}= \hat{H}_0 + \hat{H}_{SO} = -\frac{\hbar^2}{2\mu}\hat{\nabla}^2 + \frac{e^2}{4\pi\epsilon \hat{r}} +  \xi(\hat{r})\hat{l}\cdot\hat{s}
\end{equation}
\end{tcolorbox}

\subsection{Angular Momentum Addition Theorem}

There are many literature that will do an exceedingly better job at explaining this in detail than I can if I try. Instead, here I will focus on how we can apply it. First, a definition for the afficionados among us, then an example. 

The following definition is taken from Luigi Del Debbio's course\footnote{which can be found \href{https://www2.ph.ed.ac.uk/~ldeldebb/docs/QM/lect15.pdf}{here}}:\\

\noindent\textbf{Angular Momentum Addition Theorem}:\\
The allowed values of the total angular momentum quantum number $j$, given two angular momenta corresponding to quantum numbers $j_1$ and $j_2$ are

\begin{equation}
\vert j_1-j_2\vert \leq j \leq j_1+j_2 ,
\end{equation}
with a unit integer between successive $j$'s.

Let's look at some examples that are a little simpler than what A. Huxley provided in his notes. Consider the state where $l=1, s=1/2$. The possible $j$'s are

\begin{equation}
\begin{split}
j&=\{\vert l-s\vert,\vert l-s\vert +1,\dots,l+s-1,l+s \},\\
&=\left\{\frac{1}{2},\frac{3}{2} \right\},
\end{split}
\end{equation}
and the possible $m_j$'s are
\begin{equation}
\begin{split}
\text{for}\, j=\frac{1}{2},\, m_j&=\left\{-\frac{1}{2},\frac{1}{2} \right\},\,\text{and}\\
\text{for}\,j=\frac{3}{2},\, m_j&=\left\{-\frac{3}{2},-\frac{1}{2},\frac{1}{2},\frac{3}{2} \right\}.
\end{split}
\end{equation}

This addition theorem also applies to the addition of respective types of angular momentum. We will see in the following episode how this useful addition theorem gives rise to different states due to spin in a multi-electron system.


\begin{tcolorbox}[colback=pink]
For an angular momentum, $j$, composed of angular momenta $j_1$ and $j_2$, the possible values are 
\begin{equation}
\begin{split} 
j&=\vert j_1-j_2\vert, \vert j_1-j_2\vert+1,\dots,j_1+j_2,\\
m_j&=-j,-j+1,\dots,j-1,j.
\end{split}
\end{equation}
\end{tcolorbox}
\section{An Amber of Electrons}
In the above, we have described the going-ons of a hydrogen atom. Already with just a single proton and electron, we discover some pretty interesting phenomena that a lonely electron would not experience, \textit{eg} the spin-orbit interaction. However, there is a whole periodic table of atoms, and all but save the hydrogen has multiple electrons. This leads us nicely to the physics of identical particles.

In quantum mechanics, the term `identical' (or indistinguishable) has a specific meaning. That is, for a system with identical particles (\textit{eg} an amber\footnote{not actual collective noun} of electrons), the measurable observables (\textit{eg} the energy levels) must be invariant under the exchange of labels assigned to the particles. Let's put this in practice. 

First, I will quote the version of Pauli's Exclusion Principle from \cite{eisberg} that is relevant to us:\\

\noindent \textit{A system containing several electrons must be described by an antisymmetric total eigenfunction}(wavefunction).\\

Since electrons have spin, we have two possible combinations that make an overall antisymmetric wavefunction. Let $\psi_i$ and $\chi_i$ be the $i$'th electron's spatial and spin wavefunction respectively, we have: \\

\noindent\textbf{1}. symmetric in space, antisymmetric in spin:
\begin{equation}
\Psi_A(x_1,x_2;s_1,s_2)=\frac{1}{2}\left(\psi_1(x_1)\psi_2(x_2) + \psi_1(x_2)\psi_2(x_1) \right) \left( \chi_1(s_1)\chi_2(s_2) - \chi_1(s_2)\chi_2(s_1) \right).
\label{singlet}
\end{equation}

\noindent\textbf{2}. antisymmetric in space, symmetric in spin:
\begin{align}
\Psi_A(x_1,x_2;s_1,s_2) &=\frac{1}{\sqrt{2}}\left(\psi_1(x_1)\psi_2(x_2) - \psi_1(x_2)\psi_2(x_1) \right) \chi_1(s_1)\chi_2(s_1),\label{triplet1}\\
\Psi_A(x_1,x_2;s_1,s_2) &=\frac{1}{2}\left(\psi_1(x_1)\psi_2(x_2) - \psi_1(x_2)\psi_2(x_1) \right) \left( \chi_1(s_1)\chi_2(s_2) + \chi_1(s_2)\chi_2(s_1) \right),\label{triplet2}\\
\Psi_A(x_1,x_2;s_1,s_2) &=\frac{1}{\sqrt{2}}\left(\psi_1(x_1)\psi_2(x_2) - \psi_1(x_2)\psi_2(x_1) \right) \chi_1(s_2)\chi_2(s_2)\label{triplet3}.
\end{align}
In the notation above, Equation \ref{singlet} is what we call a \textit{singlet} state and Equations \ref{triplet1}-\ref{triplet3} are called the \textit{triplet} states. The subscript `A' stands for antisymmetric. That is, it has the following properties

\begin{equation}
\Psi_A(x_1,x_2;s_1,s_2) = - \Psi_A(x_2,x_1;s_2,s_1).
\end{equation}
However, as mentioned in the second paragraph, the observables are invariant under the exchange of the labels $\{x_i,s_i\}$. I encourage you to verify this by swapping the labels for a simple obserable, \textit{eg} the probability density $\vert \Psi_A\vert^2$.

Here comes my favourite aspect of identical particles.  Consider what happens to the probability density when two identical particles approach each other in the box. There are two scenarios again, which I write with a $\pm$ below:

\begin{equation}
\begin{split}
\lim_{x_1,x_2\rightarrow x} \vert\Psi_A(x_1,x_2;s_1,s_2)\vert^2 &= \frac{1}{2}\lim_{x_1,x_2\rightarrow x} \left(\psi^*_1(x_1)\psi^*_2(x_2)\pm\psi^*_1(x_2)\psi^*_2(x_1) \right)\cdot \left(\psi_1(x_1)\psi_2(x_2)\pm\psi_1(x_2)\psi_2(x_1) \right),\\
&= \frac{1}{2}\lim_{x_1,x_2\rightarrow x} \left( \vert\psi_1(x_1)\vert^2 \vert\psi_2(x_2)\vert^2 + \vert\psi_1(x_2)\vert^2 \vert\psi_2(x_1)\vert^2  \right.\\
&\quad\quad\quad\quad\quad\quad\quad\quad\left. \pm ( \psi_1^*(x_1)\psi_1(x_2)\psi_2^*(x_2)\psi_2(x_1) + \psi^*_1(x_2)\psi_1(x_1)\psi^*_2(x_1)\psi_2(x_2) ) \right),\\
&= \frac{1}{2}\lim_{x_1,x_2\rightarrow x} \left( \vert\psi_1(x_1)\vert^2 \vert\psi_2(x_2)\vert^2 + \vert\psi_1(x_2)\vert^2 \vert\psi_2(x_1)\vert^2  \right.\\
&\quad\quad\quad\quad\quad\quad\quad\quad\left. \pm (\vert\psi_1(x_1)\vert^2 \vert\psi_2(x_2)\vert^2 + \vert\psi_1(x_2)\vert^2 \vert\psi_2(x_1)\vert^2 \right),\\
\end{split}
\end{equation}
\footnote{The keen learner may ask if it is reasonable to exclude the spin component in the equation above. Since the probability density appears in an integration over \textbf{spatial} variables, we can expect the spins to orthonormalise appropriately.}I hope it is clear the $\pm$ corresponds to the singlet and triplet states respectively. Now, consider the consequence. For the case of the singlet, we have a `+', and so the probability density is enhanced by the cross terms by a factor of 2. In the case of the triplet states, the cross terms reduce the probability density to zero as the electrons approach each other.

The interpretation is clear yet profound. When the spins are anti-aligned (singlet state), there exists an \textit{attractive} force that enhances the probability density - so we can expect to \textbf{find electrons with `opposite' spins clumping together}. On the other hand, if the two electrons have the same alignment, there exists a \textit{repulsive} force such that the probability density vanishes as they coincide. This means, statistically, \textbf{electrons with the `same' spin tend to stay apart}. Remember, we haven't introduced Coulomb interaction at this stage - so this attractive/repulsive force arises from the fact that these are identical particles!

That's enough chat about identical particles for now. In the following, let's look at two most obvious interactions we can think of for a system with two electrons.
\todo[inline]{TODO: add spin angular momentum to show triplet and singlet state}
\subsection{Nuclear-Electron Coulomb Attraction}\label{coulomb attraction}

Simply put: we have a nucleus that has a net positive charge, and we have electrons that are negatively-charged - they will feel a Coulomb attraction from each other. We can quantify this interaction in the following Hamiltonian

\begin{equation}
H_0=\frac{Ze^2}{4\pi\epsilon_0r_1} + \frac{Ze^2}{4\pi\epsilon_0r_2},
\label{coulomb attraction hamiltonian}
\end{equation}
where $r_i$ corresponds to the $i$'th electron's distance  from the nucleus. 

A moment's glance and you will recognise each term to be just the usual Coulomb potential term. This time, we have $Z=2$ for a nucleus with two protons. We can write this nucleus-electron interaction as a sum (as opposed to a product) because the Coulomb attraction experienced by each electron is independent of the other. In other words, whatever force electron 1 feels due to the nucleus does not in any way affect what electron 2 experiences. 

In physics lingo, we say this interaction is \textbf{separable}. That is, the interactions of each electrons can be neatly separated out into a sum(or products). In the following, we will meet an interaction which is not trivially separable.

\begin{tcolorbox}[colback=pink]
Hamiltonian for the Nuclear-Electron Coulomb Interaction
\begin{equation}
H_0=\frac{Ze^2}{4\pi\epsilon_0r_1} + \frac{Ze^2}{4\pi\epsilon_0r_2}.
\end{equation}
\end{tcolorbox}

\subsection{Electron-Electron Coulomb Repulsion}

In a similar fashion to \S\ref{coulomb attraction}, the electrons can feel the charge of their fellow neighbours and, since they have identical charges, experience a Coulomb repulsion. We can write this interaction as

\begin{equation}
\begin{split}
H_1&=\sum_{i>j} \frac{e^2}{4\pi\epsilon_0\vert r_i-r_j\vert}, \quad (\geq \text{2 electrons})\\
&= \frac{e^2}{4\pi\epsilon_0\vert r_1-r_2\vert} \quad (\text{2 electrons}).
\end{split}
\end{equation}
For more than two electrons we have a concise expression in the first line. The summation $\sum_{i>j}$ ensures that we do not double count the number of interaction terms since the Coulomb repulsion experienced by electron $i$ due to electron $j$ is exactly identical to that felt by electron $j$ due to electron $i$.

Suppose we treat $H_1$ as a perturbation to the system described by the Hamiltonian in Equation \ref{coulomb attraction hamiltonian}. Let's see how the energy shifts in the following example cases. In the following, the notation $\bra{\alpha_1}\bra{\beta_1} \ket{\alpha_2}\ket{\beta_2} = \bra{\alpha_1}\ket{\alpha_2}\bra{\beta_1}\ket{\beta_2}$, \textit{ie} the leftmost bra acts with the leftmost ket and so forth.\\

\noindent\textbf{1}. The Ground State\\

In the ground state, the spin of the electrons must be anti-aligned. So we use the singlet expression as in Equation \ref{singlet}. We can quantify the energy shift on the ground state (1s) energy as:

\begin{equation}
\begin{split}
\Delta E&=\mel{1s,1s}{\hat{H}_1}{1s,1s},\\
&= \frac{e^2}{4\pi\epsilon_0}\mel{1s,1s}{\frac{1}{\vert \hat{r}-\hat{r}'\vert}}{1s,1s},\\
&=\frac{e^2}{4\pi\epsilon_0}\cdot \frac{1}{2}\int d^3r_1d^3r_2 (\bra{1s}\ket{r_2}\bra{1s}\ket{r_1} + \bra{1s}\ket{r_2}\bra{1s}\ket{r_1})\frac{1}{\vert r_1-r_2\vert}(\bra{r_1}\ket{1s}\bra{r_2}\ket{1s} + \bra{r_1}\ket{1s}\bra{r_2}\ket{1s}),\\
&=\frac{e^2}{4\pi\epsilon_0}\cdot \frac{1}{2}\int d^3r_1d^3r_2 (\psi^*_{1s}(r_2)\psi^*_{1s}(r_1) + \psi^*_{1s}(r_2)\psi^*_{1s}(r_1))\frac{1}{\vert r_1-r_2\vert}(\psi_{1s}(r_1)\psi_{1s}(r_2) + \psi_{1s}(r_1)\psi_{1s}(r_2)),\\
&=\frac{e^2}{4\pi\epsilon_0}\cdot 2\int d^3r_1d^3r_2  \frac{\vert\psi_{1s}(r_1)\vert^2\vert\psi_{1s}(r_2)\vert^2}{\vert r_1-r_2\vert},\\
\end{split}
\end{equation}
where in the third line I have used the completeness relation, $\mathbb{1}=\int d^3r \ket{r}\bra{r}$ and the factor of 2 in the final line comes from both electron being in the same state $n=1, L=0$ state.

Now, you might ask: have I forgotten the spin part? To which I will answer: the interaction Hamiltonian, $H_1$ depends only on the separation distance between the electrons, \textit{i.e} it doesn't care whether the particle has spin or not. Thus, the spin part just normalises to unity, which I encourage you to check.

To evaluate the integral, we use the hydrogen wavefunction with $n=1, L=0$ as we have shown in Problem Sheet 7.\\

\noindent\textbf{2}. First Excited State

In the first excited state, things become more interesting. As an electron is promoted to the $n=2$ state, it can enter either the s- or the p-shell, where $L=0, 1$ respectively. Since both electrons are in different states, we are now confronted with the option of a singlet or triplet state again since the spins can be either parallel or anti-parallel. In the following, I will denote $+/-$ on the spatial wavefunctions for the singlet and triplet states respectively. \\

\noindent\textbf{2.1}. Excited Electron in 2s-shell

Both electrons are in the s-shell. So, $l_1=l_2=0 \Rightarrow L=0$. Again, ignoring the spin component for this Hamiltonian:

\begin{equation}
\begin{split}
\Delta E(L=0) &=\mel{1s,2s}{\hat{H}_1}{1s,2s},\\
&=\frac{e^2}{4\pi\epsilon_0}\cdot \frac{1}{2}\int d^3r_1d^3r_2 (\bra{1s}\ket{r_2}\bra{2s}\ket{r_1} \pm \bra{2s}\ket{r_2}\bra{1s}\ket{r_1})\frac{1}{\vert r_1-r_2\vert}(\bra{r_1}\ket{1s}\bra{r_2}\ket{2s} \pm \bra{r_1}\ket{2s}\bra{r_2}\ket{1s}),\\
&=\frac{e^2}{4\pi\epsilon_0}\cdot \frac{1}{2}\int d^3r_1d^3r_2 (\psi^*_{1s}(r_2)\psi^*_{2s}(r_1) + \psi^*_{2s}(r_2)\psi^*_{1s}(r_1))\frac{1}{\vert r_1-r_2\vert}(\psi_{1s}(r_1)\psi_{2s}(r_2) + \psi_{2s}(r_1)\psi_{1s}(r_2)),\\
&=\frac{e^2}{4\pi\epsilon_0}\cdot \frac{1}{2}\int d^3r_1d^3r_2 
\frac{\overbrace{\vert\psi_{1s}(r_1)\vert^2\vert\psi_{2s}(r_2)\vert^2}^{\text{direct term}} \pm \overbrace{\psi_{1s}^*(r_2)\psi_{2s}(r_2)\psi_{1s}(r_1)\psi^*_{2s}(r_1)}^{\text{exchange term}}}{\vert r_1-r_2\vert}.
\end{split}
\end{equation}
The first term, labelled the `direct term', together with the $e^2$ out front, gives us the charge density. However, we pick up an additional term as well, which I have labelled the `exchange term'. Once again, the very fact that we are dealing with identical particles has given rise to an enhancement/suppression of the charge density due to the singlet/triplet state.

So, you see, even though $H_1$ is `blind' to the fact that electrons carry spin, the intrinsic quality of spin in electrons helps us to distinguish them when we observe the energy spectrum!

\noindent\textbf{2.2}. Excited Electron in 2p-shell

Now, suppose one of the electron is promoted to the 2p shell. So $l_1=0, l_2=1 \Rightarrow L=1$. Algebraically, having an electron in the 2p-shell gives an energy shift of the same form, so we skip to the final line:

\begin{equation}
\begin{split}
\Delta E(L=1) &=\frac{e^2}{4\pi\epsilon_0}\cdot \frac{1}{2}\int d^3r_1d^3r_2 
\frac{\overbrace{\vert\psi_{1s}(r_1)\vert^2\vert\psi_{2p}(r_2)\vert^2}^{\text{direct term}} \pm \overbrace{\psi_{1s}^*(r_2)\psi_{2p}(r_2)\psi_{1s}(r_1)\psi^*_{2p}(r_1)}^{\text{exchange term}}}{\vert r_1-r_2\vert}.
\end{split}
\end{equation}

What is the difference? Well, look at the exchange term. For each integration, it is a product of two different wavefunctions, $\psi_{1s}\psi_{2p}$. Since $\vert\psi_{1s}\vert^2$ overlaps more with $\vert\psi_{2s}\vert^2$ than $\vert\psi_{2p}\vert^2$ (see plot below), this convolution(aka product) means that the contribution from the exchange term is greater in the $L=0$ than $L=1$. 


\todo[inline]{TODO:Put a plot of radial wavefunction}


\begin{tcolorbox}[colback=pink]
Hamiltonian for the Electron-Electron Coulomb Interaction
\begin{equation}
H_1=\sum_{i>j} \frac{e^2}{4\pi\epsilon_0\vert r_i-r_j\vert}.
\end{equation}
\end{tcolorbox}



\newpage
\bibliographystyle{ieeetr} %the command ieeetr sorts citations in order  of appearance
\bibliography{qmBib}


\end{document}

