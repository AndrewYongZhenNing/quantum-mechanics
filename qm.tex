%\documentclass[10pt]{iopart}
\documentclass{article}
\usepackage{geometry}
\usepackage[nottoc]{tocbibind}
\usepackage{graphicx}
\usepackage{subcaption}
\usepackage{amssymb}
\usepackage{amsmath}
\usepackage{amsthm}
\usepackage{slashed}
\usepackage{bbold}
\usepackage{bm}
\usepackage{hyperref}
\usepackage{listings}
\usepackage[percent]{overpic}
\usepackage{cite}
\usepackage{yfonts}
\usepackage{physics}
\usepackage{todonotes}
\usepackage{caption}
\usepackage{subcaption}
%\usepackage{axodraw4j} %for Feynman diagrams

%Uncomment next line if AMS fonts required
%\usepackage{iopams}
\numberwithin{equation}{section} %allows numbering of equations to include section number
\usepackage[utf8x]{inputenc}
\geometry{a4paper, total={6in, 8in}}

\hypersetup{
    colorlinks=true,
    linkcolor=blue,
    filecolor=magenta,      
    urlcolor=red,
    citecolor=purple   
}

\allowdisplaybreaks

\begin{document}


\begin{titlepage}
\begin{center}
\vspace*{1.5cm}

{\large Let's Chat About \\ \vspace{0.25cm}\LARGE \textit{Quantum Mechanics}}\\
\vspace{3cm}
Andrew Zhen Ning Yong\\
\vspace{0.25cm}

Department of Physics \& Astronomy\\
University of Edinburgh

\vspace{7cm}


\end{center}
\end{titlepage}

\tableofcontents

\newpage

\section{Preamble}
First, something of an explanation and a disclaimer regarding this document.

This is my attempt at writing an informal discussion based on the content in the PHYS09053/09051 Quantum Mechanics course by C. Stock and A. Huxley. Needless to say, at best, this document should be read \textit{in complement with} the lecture notes provided, rather than instead of. 

One final thing before we start: I am prone to careless mistakes, especially where signs and numerical factors are concerned. So, if you notice anything suspicious with the expressions detailed below, please send me an email or send a pull request to my git repository.\\

\noindent Now let's chat about Quantum Mechanics.

\section{An Electron and Its Proton Pal}
\subsection{Spin-Orbit Interaction}
\todo[inline]{To be written up}

\section{An Amber of Electrons}
In the above, we have described the going-ons of a hydrogen atom. Already with just a single proton and electron, we discover some pretty interesting phenomena that a lonely electron would not experience, \textit{eg} the spin-orbit interaction. However, there is a whole periodic table of atoms, and all but save the hydrogen has multiple electrons. This leads us nicely to the physics of identical particles.

In quantum mechanics, the term `identical' (or indistinguishable) has a specific meaning. That is, for a system with identical particles (\textit{eg} an amber\footnote{not actual collective noun} of electrons), the measurable observables (\textit{eg} the energy levels) must be invariant under the exchange of labels assigned to the particles. Let's put this in practice. 

Consider the Schr\"odinger equation for a system of two non-interacting particles with spin $s_1$ and $s_2$ in a box:

\begin{equation}
-\frac{\hbar}{2m}\left( \frac{\partial^2}{\partial x_1^2} + \frac{\partial^2 }{\partial x_2^2}\right)\Psi(x_1,s_1;x_2,s_2) + V(x_1,x_2)\Psi(x_1,s_1;x_2,s_2) = E\Psi(x_1,s_1;x_2,s_2),
\label{multi particle TISE}
\end{equation}
where $\Psi(x_1,s_1;x_2,s_2)$ is the wavefunction representing the two identical particles. Now we may ask: what form could the wavefunction $\Psi$ take? Keeping in mind that observables should \textit{not} be altered by an exchange of labels (\textit{ie} swapping spatial coordinates), there are two possible solutions to Equation \ref{multi particle TISE}:

\begin{equation}
\begin{split}
\Psi_S(x_1,s_1;x_2,s_2) &= \frac{1}{\sqrt{2}} \left( \psi_{1}(x_1,s_1)\psi_{2}(x_2,s_2) + \psi_{1}(x_2,s_2)\psi_{2}(x_1,s_1)\right),\\
\Psi_A(x_1,s_1;x_2,s_2) &=\frac{1}{\sqrt{2}} \left( \psi_{1}(x_1,s_1)\psi_{2}(x_2,s_2) - \psi_{1}(x_2,s_2)\psi_{2}(x_1,s_1)\right),\\
\end{split} 
\end{equation}
where $\psi$ are the wavefunctions of the individual particles and the subscripts $`S'$ and $`A'$ stand for symmetric and antisymmetric respectively. They are called as such because of the following property

\begin{equation}
\Psi_{S/A}(x_1,s_1;x_2,s_2) = \pm \Psi_{S/A}(x_2,s_1;x_1,s_2).
\end{equation}
Note that while $\Psi_{S/A}$ picks up different signs under exchange of $x_i$ or $s_i$, observables associated to this wavefunction does not. For example, check for yourself that the measurement of the probability density function, $\vert \Psi \vert^2$ is invariant under exchange of coordinate or spin labels.

Here, I will quote the version of Pauli's Exclusion Principle from \cite{eisberg} that is relevant to us:\\

\noindent \textit{A system containing several electrons must be described by an antisymmetric total eigenfunction}(wavefunction).\\

Since electrons have spin, we have two possible combinations that make an overall antisymmetric wavefunction. Using the bra-ket notation for brevity, we have: \\

\noindent\textbf{1}. symmetric in space, antisymmetric in spin:
\begin{equation}
\ket{\Psi_A}=\frac{1}{2}\left(\ket{x_1,x_2} + \ket{x_2,x_1}  \right) \left( \ket{\uparrow,\downarrow}-\ket{\downarrow,\uparrow} \right).
\label{singlet}
\end{equation}

\noindent\textbf{2}. antisymmetric in space, symmetric in spin:
\begin{align}
\ket{\Psi_A}&=\frac{1}{\sqrt{2}}\left(\ket{x_1,x_2} - \ket{x_2,x_1}  \right) \ket{\uparrow,\uparrow},\label{triplet1}\\
\ket{\Psi_A}&=\frac{1}{2}\left(\ket{x_1,x_2} - \ket{x_2,x_1}  \right) \left( \ket{\uparrow,\downarrow}+\ket{\downarrow,\uparrow} \right),\label{triplet2}\\
\ket{\Psi_A}&=\frac{1}{\sqrt{2}}\left(\ket{x_1,x_2} - \ket{x_2,x_1}  \right) \ket{\downarrow,\downarrow}\label{triplet3}.
\end{align}
Since we are dealing with electrons, I labelled the spin part with $\uparrow$ and $\downarrow$, meaning spin-up and down respectively. In the notation above, Equation \ref{singlet} is what we call a \textit{singlet} state and Equations \ref{triplet1}-\ref{triplet3} are called the \textit{triplet} states.

Here comes my favourite aspect of identical particles. Let us continue with the case of two electrons in a box with no interaction between them. Consider what happens to the probability density when two identical particles approach each other in the box. There are two scenarios again, which I write with a $\pm$ below:

\begin{equation}
\begin{split}
\lim_{x_1\rightarrow x_2} \vert\braket{\Psi_A}\vert^2 &= \frac{1}{2}\lim_{x_1,x_2\rightarrow x} \left(\bra{x_1,x_2}\pm\bra{x_2,x_1} \right)\cdot \left(\ket{x_1,x_2}\pm\ket{x_2,x_1} \right),\\
&= \frac{1}{2}\lim_{x_1,x_2\rightarrow x} \left( \bra{x_1,x_2}\ket{x_1,x_2} + \bra{x_2,x_1}\ket{x_2,x_1} \right.\\
&\quad\quad\quad\quad\quad\quad\quad\quad\left. \pm ( \bra{x_1,x_2}\ket{x_2,x_1} + \bra{x_2,x_1}\ket{x_1,x_2} ) \right),\\
&= \frac{1}{2}\left( \bra{x,x}\ket{x,x} + \bra{x,x}\ket{x,x} \right.\\
&\quad\quad\quad\quad\left. \pm ( \bra{x,x}\ket{x,x} + \bra{x,x}\ket{x,x} ) \right).
\end{split}
\end{equation}
\footnote{The keen learner may ask if it is reasonable to exclude the spin component in the equation above. Since the probability density appears in an integration over \textbf{spatial} variables, we can expect the spins to orthonormalise appropriately.}I hope it is clear the $\pm$ corresponds to the singlet and triplet states respectively. Now, consider the consequence. For the case of the singlet, we have a `+', and so the probability density is enhanced by the cross terms by a factor of 2. In the case of the triplet states, the cross terms reduce the probability density to zero as the electrons approach each other.

The interpretation is clear yet profound. When the spins are antialigned (singlet state), there exists an \textit{attractive} force that enhances the probability density - so we can expect to \textbf{find electrons with `opposite' spins clumping together}. On the other hand, if the two electrons have the same alignment, there exists a \textit{repulsive} force such that the probability density vanishes as they coincide. This means, statistically, \textbf{electrons with the `same' spin tend to stay apart}. Remember, we haven't introduced Coulomb interaction at this stage - so this attractive/repulsive force arises from the fact that these are identical particles!

That's enough chat about identical particles for now. In the following, let's look at two most obvious interactions we can think of for a system of a positive nucleus and two electrons.

\subsection{Nuclear Coulomb Attraction}

\subsection{Inter-electron Coulomb Repulsion}




\newpage
\bibliographystyle{ieeetr} %the command ieeetr sorts citations in order  of appearance
\bibliography{qmBib}


\end{document}

